% Options for packages loaded elsewhere
\PassOptionsToPackage{unicode}{hyperref}
\PassOptionsToPackage{hyphens}{url}
\PassOptionsToPackage{dvipsnames,svgnames,x11names}{xcolor}
%
\documentclass[
  10pt,
  a4paper,
]{article}

\usepackage{amsmath,amssymb}
\usepackage{setspace}
\usepackage{iftex}
\ifPDFTeX
  \usepackage[T1]{fontenc}
  \usepackage[utf8]{inputenc}
  \usepackage{textcomp} % provide euro and other symbols
\else % if luatex or xetex
  \usepackage{unicode-math}
  \defaultfontfeatures{Scale=MatchLowercase}
  \defaultfontfeatures[\rmfamily]{Ligatures=TeX,Scale=1}
\fi
\usepackage{lmodern}
\ifPDFTeX\else  
    % xetex/luatex font selection
\fi
% Use upquote if available, for straight quotes in verbatim environments
\IfFileExists{upquote.sty}{\usepackage{upquote}}{}
\IfFileExists{microtype.sty}{% use microtype if available
  \usepackage[]{microtype}
  \UseMicrotypeSet[protrusion]{basicmath} % disable protrusion for tt fonts
}{}
\makeatletter
\@ifundefined{KOMAClassName}{% if non-KOMA class
  \IfFileExists{parskip.sty}{%
    \usepackage{parskip}
  }{% else
    \setlength{\parindent}{0pt}
    \setlength{\parskip}{6pt plus 2pt minus 1pt}}
}{% if KOMA class
  \KOMAoptions{parskip=half}}
\makeatother
\usepackage{xcolor}
\usepackage[margin=1in]{geometry}
\setlength{\emergencystretch}{3em} % prevent overfull lines
\setcounter{secnumdepth}{5}
% Make \paragraph and \subparagraph free-standing
\makeatletter
\ifx\paragraph\undefined\else
  \let\oldparagraph\paragraph
  \renewcommand{\paragraph}{
    \@ifstar
      \xxxParagraphStar
      \xxxParagraphNoStar
  }
  \newcommand{\xxxParagraphStar}[1]{\oldparagraph*{#1}\mbox{}}
  \newcommand{\xxxParagraphNoStar}[1]{\oldparagraph{#1}\mbox{}}
\fi
\ifx\subparagraph\undefined\else
  \let\oldsubparagraph\subparagraph
  \renewcommand{\subparagraph}{
    \@ifstar
      \xxxSubParagraphStar
      \xxxSubParagraphNoStar
  }
  \newcommand{\xxxSubParagraphStar}[1]{\oldsubparagraph*{#1}\mbox{}}
  \newcommand{\xxxSubParagraphNoStar}[1]{\oldsubparagraph{#1}\mbox{}}
\fi
\makeatother


\providecommand{\tightlist}{%
  \setlength{\itemsep}{0pt}\setlength{\parskip}{0pt}}\usepackage{longtable,booktabs,array}
\usepackage{calc} % for calculating minipage widths
% Correct order of tables after \paragraph or \subparagraph
\usepackage{etoolbox}
\makeatletter
\patchcmd\longtable{\par}{\if@noskipsec\mbox{}\fi\par}{}{}
\makeatother
% Allow footnotes in longtable head/foot
\IfFileExists{footnotehyper.sty}{\usepackage{footnotehyper}}{\usepackage{footnote}}
\makesavenoteenv{longtable}
\usepackage{graphicx}
\makeatletter
\newsavebox\pandoc@box
\newcommand*\pandocbounded[1]{% scales image to fit in text height/width
  \sbox\pandoc@box{#1}%
  \Gscale@div\@tempa{\textheight}{\dimexpr\ht\pandoc@box+\dp\pandoc@box\relax}%
  \Gscale@div\@tempb{\linewidth}{\wd\pandoc@box}%
  \ifdim\@tempb\p@<\@tempa\p@\let\@tempa\@tempb\fi% select the smaller of both
  \ifdim\@tempa\p@<\p@\scalebox{\@tempa}{\usebox\pandoc@box}%
  \else\usebox{\pandoc@box}%
  \fi%
}
% Set default figure placement to htbp
\def\fps@figure{htbp}
\makeatother

\makeatletter
\@ifpackageloaded{tcolorbox}{}{\usepackage[skins,breakable]{tcolorbox}}
\@ifpackageloaded{fontawesome5}{}{\usepackage{fontawesome5}}
\definecolor{quarto-callout-color}{HTML}{909090}
\definecolor{quarto-callout-note-color}{HTML}{0758E5}
\definecolor{quarto-callout-important-color}{HTML}{CC1914}
\definecolor{quarto-callout-warning-color}{HTML}{EB9113}
\definecolor{quarto-callout-tip-color}{HTML}{00A047}
\definecolor{quarto-callout-caution-color}{HTML}{FC5300}
\definecolor{quarto-callout-color-frame}{HTML}{acacac}
\definecolor{quarto-callout-note-color-frame}{HTML}{4582ec}
\definecolor{quarto-callout-important-color-frame}{HTML}{d9534f}
\definecolor{quarto-callout-warning-color-frame}{HTML}{f0ad4e}
\definecolor{quarto-callout-tip-color-frame}{HTML}{02b875}
\definecolor{quarto-callout-caution-color-frame}{HTML}{fd7e14}
\makeatother
\makeatletter
\@ifpackageloaded{caption}{}{\usepackage{caption}}
\AtBeginDocument{%
\ifdefined\contentsname
  \renewcommand*\contentsname{Table of contents}
\else
  \newcommand\contentsname{Table of contents}
\fi
\ifdefined\listfigurename
  \renewcommand*\listfigurename{List of Figures}
\else
  \newcommand\listfigurename{List of Figures}
\fi
\ifdefined\listtablename
  \renewcommand*\listtablename{List of Tables}
\else
  \newcommand\listtablename{List of Tables}
\fi
\ifdefined\figurename
  \renewcommand*\figurename{Figure}
\else
  \newcommand\figurename{Figure}
\fi
\ifdefined\tablename
  \renewcommand*\tablename{Table}
\else
  \newcommand\tablename{Table}
\fi
}
\@ifpackageloaded{float}{}{\usepackage{float}}
\floatstyle{ruled}
\@ifundefined{c@chapter}{\newfloat{codelisting}{h}{lop}}{\newfloat{codelisting}{h}{lop}[chapter]}
\floatname{codelisting}{Listing}
\newcommand*\listoflistings{\listof{codelisting}{List of Listings}}
\makeatother
\makeatletter
\makeatother
\makeatletter
\@ifpackageloaded{caption}{}{\usepackage{caption}}
\@ifpackageloaded{subcaption}{}{\usepackage{subcaption}}
\makeatother

\usepackage{bookmark}

\IfFileExists{xurl.sty}{\usepackage{xurl}}{} % add URL line breaks if available
\urlstyle{same} % disable monospaced font for URLs
\hypersetup{
  pdftitle={Workiva Inc (WK)},
  pdfauthor={Eduardo Ramirez},
  colorlinks=true,
  linkcolor={blue},
  filecolor={Maroon},
  citecolor={Blue},
  urlcolor={Blue},
  pdfcreator={LaTeX via pandoc}}


\title{Workiva Inc (WK)}
\usepackage{etoolbox}
\makeatletter
\providecommand{\subtitle}[1]{% add subtitle to \maketitle
  \apptocmd{\@title}{\par {\large #1 \par}}{}{}
}
\makeatother
\subtitle{CFA Institute Research Challenge}
\author{Eduardo Ramirez}
\date{2025-02-10}

\begin{document}
\maketitle


\setstretch{1}
\section{Business Description}\label{business-description}

Worki va Inc.~provides cloud-based and mobile-enabled platforms for
enterprises. The Company offers software to collect, manage, report, and
analyze business data in real time. ``Workiva delivers a unified
platform that offers financial and non-financial investor-grade business
reporting, a unique and key differentiator separating us from the
competition'' (2023 annual report: CEO Julie ISkow). ``Platform for
assured integrated reporting---bringing financial reporting, ESG, and
GRC teams together in one controlled, secure, audit-ready environment
and helping them orchestrate wo rk across departments and eliminate
technology silos'' (2023 annual report).

See details in
\hyperref[industry-overview-competitive-positioning]{Industry Overview}.

\section{Snapshot}\label{snapshot}

\begin{tcolorbox}[enhanced jigsaw, colframe=quarto-callout-note-color-frame, colback=white, opacityback=0, bottomrule=.15mm, leftrule=.75mm, arc=.35mm, rightrule=.15mm, toprule=.15mm, left=2mm, breakable]

\textbf{Recommendation:} Hold\\
\emph{Price Target:} \$102 per share

\end{tcolorbox}

Although Workiva demonstrates promising long-term potential (see
Business Description and ESG sections), we recommend a Hold rating.
Slowing growth trends and persistent net losses warrant caution over the
near to mid-term, despite strong recurring revenue and improving gross
margins.

\subsection{Key Points}\label{key-points}

\subsubsection{Business Model \& Growth
Prospects}\label{business-model-growth-prospects}

\begin{itemize}
\tightlist
\item
  Cloud-Native Platform: Integrates financial, ESG, and GRC reporting in
  a unified environment.
\item
  Recurring Revenue Strength: Subscription revenues account for
  \textasciitilde89\% of total, supporting reliable cash flows (see
  Financial Analysis).
\end{itemize}

\subsubsection{Revenue Growth vs.~Potential
Deceleration}\label{revenue-growth-vs.-potential-deceleration}

\begin{itemize}
\tightlist
\item
  Steady Top-Line Growth: Revenue rose from \$351.6M (2020) to \$630.0M
  (2023). Year-over-year quarterly growth remains in the mid-teens
  (14.5--17.35\%).
\item
  Growth Slowing as WK Scales: The pace has moderated from previous
  20\%+ territory, suggesting a more measured expansion going forward
  (see Quarterly Financial Performance).
\end{itemize}

\subsubsection{Profitability Challenges}\label{profitability-challenges}

\begin{itemize}
\tightlist
\item
  Elevated SG\&A (58--62\% of Sales): High operating costs have
  contributed to persistent net losses (e.g., -\$127.5M in 2023).
\item
  Potential Margin Leverage: Gross margins hover around 77\%, indicating
  room for operating leverage if SG\&A moderates. Recent quarters (Q3 \&
  Q4 2024) showed smaller losses, but a clear path to profitability is
  not yet confirmed.
\end{itemize}

\subsubsection{Valuation \& Rationale for
Hold}\label{valuation-rationale-for-hold}

\begin{itemize}
\tightlist
\item
  Monte Carlo \& DCF Analyses: Indicate a fair value around \$102 per
  share, assuming continued growth (18--20\%) and stable margins (see
  Valuation).
\item
  Hesitation: Uncertainties around slower near-term demand and the
  company's ability to trim operating expenses justify a more
  conservative approach.
\item
  Relative Valuation: WK trades at a premium vs.~larger peers (SAP,
  Oracle, Intuit), supported by high recurring revenue and ESG/GRC
  tailwinds---but this premium might compress if growth decelerates
  significantly.
\end{itemize}

\subsubsection{ESG \& Long-Term
Advantages}\label{esg-long-term-advantages}

\begin{itemize}
\tightlist
\item
  ESG Leadership: High renewal rates (\textasciitilde98\%) reflect
  sticky customer relationships, especially in ESG/GRC solutions.
  Workiva's integrated reporting platform aligns with shifting global
  regulations and stakeholder expectations (see ESG section).
\item
  Differentiation: ESG capabilities and a unified reporting solution
  position WK well in a rapidly evolving regulatory environment.
\end{itemize}

Workiva's innovative platform and high retention rates underscore its
long-term potential, particularly as ESG and integrated reporting needs
expand. However, slowing revenue growth, persistently high SG\&A, and
consistent net losses create near-term uncertainties. While the
underlying fundamentals may support a positive long-run view, we issue a
Hold rating due to the potential for slower growth and the need for
disciplined cost management before WK can convincingly transition to
sustained profitability.

\section{Industry Overview \& Competitive
Positioning}\label{industry-overview-competitive-positioning}

\subsection{1. Threat of New Entrants:}\label{threat-of-new-entrants}

\begin{itemize}
\tightlist
\item
  \textbf{Barriers to Entry:} New entrants would face significant
  challenges complying with evolving regulations in financial reporting,
  ESG, and GRC. Developing a robust, integrated platform with strong
  reliability, performance, and audit capabilities further raises the
  entry hurdle. Building the product breadth and brand reputation
  Workiva has achieved requires significant time and investment.
\item
  \textbf{Economies of Scale:} Workiva's cloud-based approach and strong
  network effects enable it to serve more customers without a linear
  increase in costs. As its user base expands, Workiva can invest more
  in platform innovation and integrations, maintaining a competitive
  edge.
\item
  \textbf{Access to Supply Channels:} Existing relationships with
  enterprise clients and industry stakeholders function as an informal
  distribution network for Workiva's solutions. Newcomers must forge
  similar ties from scratch, limiting their initial market penetration.
\end{itemize}

\subsection{2. Bargaining Power of
Customers:}\label{bargaining-power-of-customers}

\begin{itemize}
\tightlist
\item
  \textbf{Price Takers:} Customers tend to be price takers since they
  can choose from various competitors---ranging from manual processes to
  diversified software solutions---although high switching costs and the
  need for robust global reporting moderate this power.
\item
  \textbf{Alternatives:} The availability of numerous alternatives
  (including niche and broad enterprise solutions) gives customers
  options.
\end{itemize}

\subsection{3. Bargaining Power of
Suppliers:}\label{bargaining-power-of-suppliers}

\begin{itemize}
\tightlist
\item
  \textbf{Price Takers:} Suppliers---primarily cloud infrastructure
  providers---are essentially price takers due to a competitive market
  with multiple viable alternatives.
\item
  \textbf{Supply Constraints:} The competitive nature of the cloud
  computing market ensures that no single supplier can impose
  significant constraints, thereby limiting their bargaining power.
\end{itemize}

\subsection{4. Threat of Substitute Products or Services:
High}\label{threat-of-substitute-products-or-services-high}

\begin{itemize}
\tightlist
\item
  While customers can turn to various substitutes---from traditional
  manual methods to specialized software---Workiva's unique integrated
  platform and robust data-linking capabilities make it a challenging
  alternative, though the threat of substitution remains moderately
  high.
\end{itemize}

\subsection{5. Intensity of Competitive
Rivalry}\label{intensity-of-competitive-rivalry}

\begin{itemize}
\tightlist
\item
  Workiva faces intense competitive rivalry from a wide range of
  competitors including diversified enterprise software providers, niche
  vendors, and even internal reporting solutions, all contending on
  product features, price, and innovation in a rapidly evolving market.
\end{itemize}

\section{Financial Analysis}\label{financial-analysis}

Workiva Inc.~has demonstrated robust revenue expansion both annually and
quarterly. Annual revenue climbed from \$351.6 million in 2020 to
\$630.0 million in 2023, with year-over-year growth decelerating from
26.1\% in 2021 to 17.1\% in 2023. In parallel, quarterly performance in
2024 shows a steady acceleration: quarter-over-quarter revenue growth
improved from 1.04\% in Q2 to 7.69\% in Q4, while year-over-year
quarterly growth rates ranged between 14.50\% and 17.35\%. This
consistent top-line momentum is primarily driven by the subscription and
support revenue stream, which underpins Workiva's recurring revenue
model.

Despite these strong growth metrics, profit concerns persist. Elevated
SG\&A expenses, which have remained in the high 50\% to low 60\% range
of revenue in recent quarters, have contributed to sustained operating
and net losses. Although recent trends in Q3 and Q4 2024 suggest a
potential easing of losses after a spike in Q2, overall profitability
remains a challenge. For investors, the key lies in monitoring whether
the company's significant investments in growth can eventually yield the
operating leverage needed to reverse the current loss trajectory.

\subsection{Revenue Growth vs.~Profitability
Comparison}\label{revenue-growth-vs.-profitability-comparison}

\begin{itemize}
\tightlist
\item
  \textbf{Revenue trends:} 19\% CAGR from \$443.3M (2021) to \$630M
  (2023).
\item
  Workiva has had steady growth over the years as shown in data 1 table.
  The compound annual growth rate of revenue was 20\% from 2018 to 2020,
  22\% from 2019 to 2021, and 24\% from 2020 to 2022. From 2021 to 2023,
  the compound annual growth rate was 19\%.
\item
  The year-over-year percentage increase in total revenue was 17.5\% in
  2018, 21.9\% in 2019, 18.0\% in 2020, 26.1\% in 2021, 21.3\% in 2022,
  and 17.1\% in 2023.
\item
  Approximately 84\% of the revenue in 2020, 86\% of the revenue in
  2021, 86\% of the revenue in 2022, and 89\% of the revenue in 2023 was
  derived from subscription and support fees, with the remainder from
  professional services.
\item
  \textbf{Profitability concerns:} Rising net losses (\$127.5M in 2023).
  Workiva has incurred net losses in recent years as shown in the data
  1.1 table.
\item
  \textbf{Subscription revenue dominance:} 89\% of total revenue.
\end{itemize}

\subsection{Quarterly Financial Performance Analysis
(2019-2024)}\label{quarterly-financial-performance-analysis-2019-2024}

\subsubsection{1. Revenue Analysis}\label{revenue-analysis}

\begin{itemize}
\tightlist
\item
  While revenue consistently increases, the rate of QoQ growth varies as
  shown in data 2 table.
\item
  \textbf{Year-over-Year (YoY) Revenue Growth:} To get a clearer picture
  of sustained growth, let's look at YoY growth in the data 3 table.
  These YoY growth rates, while still positive, are in the mid-teens,
  and seem to be slightly lower than the historical annual growth rates
  mentioned in the year report (which were in the 20\%+ range for
  earlier periods). This potentially suggests a deceleration in the
  revenue growth rate as the company scales, aligning with the annual
  report commentary.
\item
  \textbf{Implications and Link to Annual Report:} The quarterly data
  reinforces the annual report observation that while revenue growth is
  still robust, the percentage increase is showing signs of slowing
  down. This is a typical pattern for maturing growth companies.
  However, sustaining double-digit YoY quarterly growth consistently is
  still a positive signal.
\end{itemize}

\subsubsection{2. Selling, General and Administrative (SG\&A) Expense
Analysis}\label{selling-general-and-administrative-sga-expense-analysis}

\begin{itemize}
\tightlist
\item
  \textbf{Increasing SG\&A Expenses:} SG\&A expenses have also generally
  increased over the quarters, from \$35.748 million in Q1 2019 to
  \$116.913 million in Q4 2024. This is not unexpected for a growing
  company investing in sales, marketing, and infrastructure to support
  future expansion.
\item
  \textbf{SG\&A as a Percentage of Revenue:} The SG\&A as a percentage
  of revenue appears to be fluctuating in recent quarters but with a
  general trend of staying in the high 50s to low 60s percentage range.
  There is a slight dip in Q4 2024, which could be a positive sign if
  sustained. However, overall, SG\&A is consuming a significant portion
  of revenue as shown in data 4 table.
\item
  \textbf{Implications:} The increasing SG\&A costs, especially as a
  percentage of revenue remains high, are important to monitor. For a
  SaaS company like Workiva (implied by subscription revenue focus in
  the annual report), the goal is often to achieve operating leverage
  where revenue growth outpaces SG\&A growth over time, leading to
  improved profitability. Currently, the data suggests that while
  revenue is growing, SG\&A is growing almost in proportion, preventing
  significant operating leverage gains in these periods.
\end{itemize}

\subsubsection{3. Profitability Analysis}\label{profitability-analysis}

\begin{itemize}
\tightlist
\item
  \textbf{Operating Losses:} Workiva has consistently reported operating
  losses across all quarters analyzed. The operating loss trend is
  somewhat volatile: While still negative, the operating losses seem to
  be decreasing in the most recent two quarters (Q3 \& Q4 2024) after a
  spike in Q2 2024. This could indicate improving operating efficiency
  or expense management in the latter half of 2024, but it's too early
  to confirm a trend reversal based on just two quarters.
\item
  \textbf{Net Losses Available to Common Shareholders:} The net losses
  mirror the operating loss trend, and Workiva has consistently been net
  loss-making throughout the period. Similar to operating losses, the
  net losses also show a potential decrease in the most recent two
  quarters of 2024. However, the net loss in Q4 2024 is still higher
  than in Q1 2019, indicating that overall profitability has not yet
  been achieved.

  \begin{itemize}
  \tightlist
  \item
    The quarterly losses are consistent with the annual report
    information highlighting net losses for the full years (2020-2023).
    The annual report also mentioned an ``induced conversion expense''
    in 2023 which significantly impacted net loss in that year. Without
    knowing if similar one-time expenses are impacting these quarterly
    results, it's harder to isolate the underlying operational
    profitability trend just from this data.
  \end{itemize}
\end{itemize}

\subsubsection{4. Gross Margin Analysis}\label{gross-margin-analysis}

Workiva's gross margin has generally trended upward over time as we can
see in data 5.1 table, reflecting increasing operating efficiency in
delivering its subscription-based services:

\begin{itemize}
\tightlist
\item
  \textbf{2019--2020 Transition:} Gross margin improved from the
  low-70\% range (e.g., 72.1\% in Q1 2019) to the mid-70\% range by late
  2020.
\item
  \textbf{2021--2022 Levels:} It remained relatively stable in the
  75--76\% range, indicating that economies of scale and cost management
  are holding steady.
\item
  \textbf{2023--2024 Improvements:} Gross margin in Q4 2023 reached
  approximately 77.3\%, and Q4 2024 ended near 77.1\%, among the highest
  levels in the dataset.

  \begin{itemize}
  \tightlist
  \item
    A mid- to high-70\% gross margin is strong for a SaaS-oriented
    business. However, elevated SG\&A spending still outweighs these
    gains, preventing net profitability. Maintaining or expanding these
    margins will be critical to Workiva's path toward net income
    positivity.
  \end{itemize}
\end{itemize}

\subsubsection{5. Cash Flow Analysis}\label{cash-flow-analysis}

\begin{itemize}
\tightlist
\item
  \textbf{Fluctuating Cash Balances:} Cash has swung significantly as
  shown in data 5.2 table---rising sharply in certain quarters (e.g., Q3
  2019, Q3 2023) and then declining, suggesting periodic capital raises
  or debt issuances to fund operations. Despite ongoing net losses,
  Workiva has generally maintained adequate cash to cover near-term
  obligations.
\item
  \textbf{Liquidity Ratios:} Current and Quick Ratios typically stay
  above 1.0, indicating that short-term assets exceed short-term
  liabilities. Cash Ratio often hovers around 0.5--0.9, dipping below
  1.0 unless bolstered by a capital raise. (see appendix figure 5.2.1
  for Current and Cash ratio comparison)
\item
  \textbf{High Days' Sales in Receivables:} High Days' Sales in
  Receivables (often above 200 days) can delay cash collections,
  reinforcing the importance of efficient receivables management.
\end{itemize}

\subsubsection{6. Key Observations}\label{key-observations}

\begin{itemize}
\tightlist
\item
  \textbf{Growth Company in Investment Phase:} Workiva exhibits
  characteristics of a growth company that is still heavily investing
  for future expansion. Consistent revenue growth is a positive
  indicator of market demand and product acceptance.
\item
  \textbf{Revenue Growth Deceleration Needs Monitoring:} While YoY and
  QoQ revenue growth is positive, the percentage growth rates seem to be
  moderating. It is crucial to monitor if this is a natural deceleration
  as the company scales or a sign of slowing market momentum. Sustaining
  a healthy double-digit growth rate remains important.
\item
  \textbf{High SG\&A Costs Impacting Profitability:} SG\&A expenses are
  consuming a significant portion of revenue, hindering the path to
  profitability. Workiva needs to demonstrate operating leverage by
  growing revenue at a faster rate than SG\&A expenses to improve
  profitability metrics over time. The recent slight dip in SG\&A as a
  percentage of revenue in Q4 2024, if sustained, could be an early
  positive signal.
\item
  \textbf{Operating and Net Losses Persist, but Recent Improvement
  Possible:} Workiva is still operating at a loss, which is not uncommon
  for growth-oriented companies. However, the recent quarterly data (Q3
  \& Q4 2024) suggests a potential trend of decreasing operating and net
  losses, which warrants close observation in future quarters. It is too
  early to definitively say if the company is on a clear path to
  profitability based on just two quarters of data.
\end{itemize}

\section{Financial Analysis
Conclusion}\label{financial-analysis-conclusion}

Workiva is demonstrating revenue growth, but profitability remains a
challenge. The quarterly data suggests a potential stabilization or even
slight improvement in loss trends in the most recent periods. However,
continued focus on operating leverage, efficient SG\&A management, and
sustaining healthy revenue growth, particularly in subscription revenue,
will be critical for Workiva to achieve and maintain profitability and
enhance shareholder value. Further monitoring of these quarterly trends
and deeper analysis as recommended above are essential for a more
comprehensive CFA-style financial assessment.

\section{Valuation}\label{valuation}

\subsection{Valuation methodologies:}\label{valuation-methodologies}

\begin{itemize}
\tightlist
\item
  Incorporated Monte Carlo simulations using growth (18--20\%), gross
  margin (\textasciitilde77\%), SG\&A (\textasciitilde61\%), and renewal
  rate (\textasciitilde98\%). Discounted future free cash flows,
  balancing short-term net losses with anticipated operating leverage
  and margin expansion.
\item
  Relative valuation

  \begin{itemize}
  \tightlist
  \item
    EV/Revenue \& P/S: Benchmarked vs.~SAP, Oracle, Intuit, and
    AuditBoard; Workiva trades at a premium justified by higher
    recurring revenue mix (89\%) and robust ESG/GRC exposure.
  \item
    EV/EBITDA: Currently negative due to net losses, but improving
    trajectory suggests multiple re-rating as SG\&A scales and operating
    leverage improves.
  \end{itemize}
\item
  Comparable company analysis vs.~SAP, Oracle, Intuit, AuditBoard.

  \begin{itemize}
  \tightlist
  \item
    Growth-centric peers (Intuit, AuditBoard) support higher multiples;
    large-cap peers (SAP, Oracle) offer established baseline.
  \item
    Workiva's competitive edge lies in its integrated reporting
    ecosystem, ESG/GRC leadership, and near-98\% renewal rates.
  \end{itemize}
\end{itemize}

\subsection{Key valuation drivers:}\label{key-valuation-drivers}

\begin{itemize}
\tightlist
\item
  Growth in ESG \& GRC adoption.

  \begin{itemize}
  \tightlist
  \item
    Sustained demand for integrated financial and non-financial
    reporting solutions.
  \item
    Expanding suite of global statutory, SOX, and audit solutions
    fueling cross-selling.
  \end{itemize}
\item
  Subscription renewal rates (97.9\%).

  \begin{itemize}
  \tightlist
  \item
    High retention underscores sticky customer base and recurring
    revenue stability.
  \item
    Strengthens DCF assumptions for future cash flow predictability.
  \end{itemize}
\item
  Operating leverage potential.

  \begin{itemize}
  \tightlist
  \item
    Gross margin near 77\% supports scaling; SG\&A ratio
    (\textasciitilde60\%+) is expected to moderate with top-line growth.
  \item
    Room for margin expansion as product and geography investments
    mature.
  \end{itemize}
\end{itemize}

\subsection{Target price estimation \& valuation sensitivity
analysis.}\label{target-price-estimation-valuation-sensitivity-analysis.}

\begin{itemize}
\tightlist
\item
  Monte Carlo Outcome: Central scenario converges around \$102 per
  share, reflecting moderate growth (18--20\%) and stable gross margins.
\item
  Sensitivity Analysis:

  \begin{itemize}
  \tightlist
  \item
    +/- 2\% change in revenue growth alters fair value by
    \textasciitilde\$8--\$10.
  \item
    SG\&A Efficiency: 1--2\% drop in SG\&A as \% of revenue boosts
    margins and can raise valuation by mid-single-digit percentages.
  \item
    Renewal Rate Impact: Each 1\% shift in subscription renewal rate
    significantly affects recurring revenue base and forward multiples.
  \end{itemize}
\end{itemize}

\subsection{Conclusion:}\label{conclusion}

Integrating DCF, relative valuations, and competitive benchmarking
supports a \$102 target price. Upside hinges on sustained ESG/GRC
demand, high renewal rates, and disciplined cost management. Downside
risk lies in slower revenue growth or delayed operating leverage.

\section{Investment Risks}\label{investment-risks}

\subsection{Balancing Growth with Financial Stability: Workiva's Key
Risks:}\label{balancing-growth-with-financial-stability-workivas-key-risks}

Workiva's risk factors could challenge the company's ability to sustain
this momentum. These risks, which span regulatory, competitive,
operational, financial, macroeconomic, and broader industry
considerations, underscore the delicate balance between investing for
expansion and maintaining financial stability.

\begin{itemize}
\tightlist
\item
  \textbf{Regulatory risks:}

  \begin{itemize}
  \tightlist
  \item
    Ongoing legislative and regulatory changes remain central to
    Workiva's market. The complexity of regulations requiring
    transparent reporting is a key driver for its platform, but sudden
    shifts in disclosure rules---particularly for ESG---can alter demand
    and impose new compliance burdens.
  \item
    Stricter data privacy regulations (e.g., GDPR) and country-specific
    mandates intensify the pressure to maintain consistent global
    compliance.
  \item
    As Workiva continues to expand internationally, it must navigate
    diverse legal and regulatory requirements, which could elevate
    operating costs if different regions impose overlapping or
    conflicting standards.
  \end{itemize}
\item
  \textbf{Competitive risks:}

  \begin{itemize}
  \tightlist
  \item
    The competitive landscape includes large enterprise software
    providers, which may leverage broader product portfolios to undercut
    pricing or bundle solutions. This risk intensifies as major vendors
    invest more aggressively in cloud-based platforms that overlap with
    Workiva's offerings.
  \item
    Smaller niche players can gain traction quickly by focusing on
    specific solutions and potentially luring customers with lower entry
    costs.
  \item
    Given that Workiva operates in a market where competing solutions
    evolve rapidly, sustaining a technological edge in areas like data
    linking, collaboration features, and integrations is critical to
    defending its market position.
  \end{itemize}
\item
  \textbf{Operational risks:}

  \begin{itemize}
  \tightlist
  \item
    Cybersecurity threats continue to escalate across the software
    industry, posing significant concerns for a platform that handles
    sensitive financial and ESG data. A breach could damage Workiva's
    reputation, disrupt services, and result in legal exposure.
  \item
    Implementing emerging technologies---such as generative AI---must be
    carefully managed to avoid system failures or integration issues
    that could erode customer trust.
  \item
    Foreign operations entail compliance with various local regulations,
    foreign exchange complexities, and geopolitical uncertainties.
    Maintaining consistent service levels across multiple regions and
    time zones requires significant coordination and resources.
  \end{itemize}
\item
  \textbf{Financial risks:}

  \begin{itemize}
  \tightlist
  \item
    Workiva's consistent net losses signal ongoing challenges with
    profitability. Elevated SG\&A expenditures, alongside investments in
    product development and international expansion, could strain
    operating margins if revenue growth decelerates.
  \item
    Rising debt levels or the need for further capital raises may be
    subject to unfavorable market conditions, thus constraining
    financial flexibility.
  \item
    Workiva's stock price is also susceptible to volatility, influenced
    by broader market trends and investor sentiment. Significant sales
    of shares or conversions of notes can add downward pressure on share
    value, further complicating capital-raising efforts.
  \end{itemize}
\item
  \textbf{Macroeconomic risks:}

  \begin{itemize}
  \tightlist
  \item
    An economic downturn could reduce enterprise software spending,
    placing pressure on Workiva's top-line growth.
  \item
    Heightened recessionary concerns or sudden shifts in interest rates
    can lead organizations to delay purchasing decisions or scale back
    SaaS subscriptions---particularly for solutions deemed
    non-essential.
  \item
    International market volatility, whether stemming from currency
    fluctuations or geopolitical conflicts, adds another layer of
    uncertainty. Adverse exchange rate movements could negatively affect
    reported financial performance.
  \end{itemize}
\item
  \textbf{Industry risk:}

  \begin{itemize}
  \tightlist
  \item
    Workiva operates in a landscape shaped by rapid technological
    change, frequent updates to industry standards, and consolidation
    among major players.
  \item
    Some competitors may opt to align their pricing aggressively or
    bundle regulatory reporting capabilities with broader enterprise
    offerings.
  \item
    Others could respond quickly to evolving demands by launching
    competing features---especially in GRC and ESG-related solutions.
  \item
    Rising cybersecurity threats represent an industry-wide challenge,
    amplifying the need for ongoing investment in secure infrastructure.
  \item
    Should Workiva fail to anticipate technology shifts, maintain robust
    partnerships, or address emerging customer needs, it risks losing
    ground in a highly dynamic environment.
  \end{itemize}
\end{itemize}

While Workiva continues to demonstrate notable momentum, balancing this
growth with prudent financial management remains a pivotal concern.
Regulatory shifts, pricing pressure from larger competitors, operational
complexities tied to cybersecurity and AI, persistent net losses,
macroeconomic headwinds, and intensifying industry competition all
represent meaningful hurdles. Overcoming these challenges will hinge on
effectively managing the pace of strategic investments, optimizing
go-to-market execution, and reinforcing the platform's value proposition
amid evolving market demands.

\section{Environmental, Social, and Governance
(ESG)}\label{environmental-social-and-governance-esg}

Workiva's commitment to ESG is woven into the fabric of its platform,
aligning financial and non-financial processes to support organizations
seeking greater transparency, accountability, and stakeholder trust. By
integrating ESG reporting alongside financial reporting and GRC, Workiva
provides a unified, audit-ready environment that simplifies data
collection, collaboration, and disclosure. This focus reflects a broader
industry shift toward stakeholder capitalism, where organizations are
increasingly expected to demonstrate responsible environmental
practices, a strong social conscience, and sound governance frameworks.

\subsection{ESG Integration: Workiva's platform supports ESG reporting
for
clients.}\label{esg-integration-workivas-platform-supports-esg-reporting-for-clients.}

\begin{itemize}
\tightlist
\item
  A notable aspect of Workiva's ESG footprint is its ESG Integration.
  The platform automates data gathering from disparate ESG sources and
  supports the preparation of investor-grade disclosures that can span
  sustainability reports, statutory filings, and earning call scripts,
  including XBRL tagging for standardized reporting.
\item
  The ESG Explorer tool helps teams reference frameworks like GRI, SASB,
  TCFD, and UNSDGs, positioning users to navigate the evolving
  regulatory landscape effectively.
\item
  GRC Connections tie controls to ESG metrics, illustrating Workiva's
  emphasis on bridging traditional compliance activities with emerging
  sustainability requirements.
\end{itemize}

\subsection{Company ESG Ratings:}\label{company-esg-ratings}

\begin{itemize}
\tightlist
\item
  Beyond enabling clients to achieve comprehensive ESG disclosures,
  Workiva has attained recognized ESG Ratings of its own. The consistent
  AAA rating from MSCI underscores the company's effective management of
  ESG risks and opportunities, elevating its position in the market.
\item
  High ratings can be a differentiating factor for analysts and
  shareholders who increasingly weigh ESG performance when assessing
  long-term growth potential and risk profiles.
\end{itemize}

\subsection{Diversity \& Inclusion}\label{diversity-inclusion}

\begin{itemize}
\tightlist
\item
  WK also prioritizes Diversity \& Inclusion (DE\&I), treating it not as
  a peripheral objective but as a fundamental pillar of its corporate
  culture.
\item
  By sponsoring Business Employee Resource Groups, recruiting from broad
  talent pools, and tracking representation goals, Workiva underscores
  its commitment to fostering belonging, empowerment, and innovation.
\item
  These inclusive initiatives---complemented by robust leadership
  sponsorship---help create an environment where diverse perspectives
  and backgrounds thrive.
\end{itemize}

\subsection{Sustainability Strategy:}\label{sustainability-strategy}

\begin{itemize}
\tightlist
\item
  Underpinning Workiva's Sustainability Strategy is an established
  governance structure, guided by an internal ESG Task Force and advised
  by an external ESG Advisory Council.
\item
  Aligning with the United Nations Sustainable Development Goals and the
  Task Force on Climate-related Disclosures, Workiva advances
  initiatives that include corporate philanthropy, environmentally
  responsible business practices, and transparent reporting.
\item
  These efforts, in turn, inform how stakeholders perceive the company's
  long-term viability and risk management approach.
\end{itemize}

\subsection{How ESG practices affect valuation \& investor
sentiment.}\label{how-esg-practices-affect-valuation-investor-sentiment.}

\begin{itemize}
\tightlist
\item
  High-quality ESG disclosures, strengthened by Workiva's platform
  capabilities, can mitigate reputational risks, meet shareholder
  expectations, and potentially enhance market confidence.
\item
  As companies face intensifying pressure to disclose both financial and
  non-financial results, the ability to deliver accurate, reliable ESG
  data may contribute to investor optimism and, by extension, support
  favorable valuations in competitive capital markets.
\end{itemize}

\subsection{Key ESG view}\label{key-esg-view}

\begin{itemize}
\tightlist
\item
  Workiva's strong ESG focus and consistently high ratings indicate
  proactive risk management and alignment with stakeholder
  expectations---factors that may appeal to socially responsible and
  traditional investors alike.
\item
  Combined with the company's unified approach to financial and
  non-financial reporting, this ESG leadership can potentially drive
  brand value and differentiate the platform in a rapidly maturing
  market.
\item
  For investors placing a premium on responsible corporate practices and
  sustained growth in an expanding ESG environment, Workiva's position
  and performance in this sphere present a compelling case.
\end{itemize}

\section{Investment Summary}\label{investment-summary}

\textbf{Recommendation: HOLD}

Workiva's cloud-native platform addresses growing regulatory and
stakeholder demands for integrated financial, ESG, and GRC reporting.
With approximately 89\% of revenue from subscription-based services and
a renewal rate near 98\%, Workiva benefits from high customer retention
and dependable recurring revenue. Its strong gross margin (around 77\%)
indicates potential for operating leverage once spending stabilizes.

\subsection{Key Considerations}\label{key-considerations}

\begin{itemize}
\tightlist
\item
  \textbf{Strong Market Tailwinds:} Heightened compliance and ESG
  requirements support continuing demand for Workiva's unified reporting
  solutions, driving solid top-line growth.
\item
  \textbf{Recurring Revenue Model:} Subscription revenue dominance
  underpins predictable cash flows and fosters a sticky customer base.
\item
  \textbf{Profitability Challenges:} Elevated SG\&A (58--62\% of sales)
  and ongoing net losses (-\$127.5M in 2023) raise concerns about
  near-term earnings. While recent quarters hint at smaller losses, a
  clear path to sustained profitability remains unproven.
\item
  \textbf{Valuation \& Risk:} Monte Carlo and DCF analyses converge
  around \$102 per share, assuming moderate revenue growth (18--20\%)
  and stable margins. However, any slowdown in growth or delays in
  reducing operating costs could pressure valuation.
\end{itemize}

\subsection{Conclusion}\label{conclusion-1}

Workiva's emphasis on compliance and ESG, coupled with strong
subscription revenues, suggests significant long-term promise.
Nevertheless, persistent losses and ongoing cost pressures warrant
caution. We recommend a HOLD, anticipating that Workiva will need to
demonstrate consistent cost discipline and a clear path to profitability
before justifying a more bullish stance.

test Figure~\ref{fig-testdel}

\begin{itemize}
\tightlist
\item
  \textbf{Differentiate between deterministic and stochastic trends}

  \begin{itemize}
  \tightlist
  \item
    Stochastic Trend: Random trend that does not follow a discernible or
    predictable pattern.
  \item
    Deterministic Trend: Can be modeled with mathematical functions,
    facilitating the long-term prediction of the behavior.
  \end{itemize}
\end{itemize}

known as {ergodic}\footnote{{A time series model that is stationary in
  the mean is ergodic in the mean if the time average for a single time
  series tends to the ensemble mean as the length of the time series
  increases (2.2 2.2.3).}}, and the models in this book are all ergodic.

\begin{figure}

\centering{

\pandocbounded{\includegraphics[keepaspectratio]{ss/testdel.png}}

}

\caption{\label{fig-testdel}}

\end{figure}%




\end{document}
